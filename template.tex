\documentclass{article}

\usepackage{fontspec}
\usepackage{amsmath}
\usepackage{listings}
\usepackage{xcolor}
\definecolor{codegreen}{rgb}{0,0.6,0}
\definecolor{codegray}{rgb}{0.5,0.5,0.5}
\definecolor{codepurple}{rgb}{0.58,0,0.82}
\definecolor{backcolour}{rgb}{0.95,0.95,0.92}


\pagestyle{fancy}
\lhead{\textbf{\CourseCode~\CoursName}~课程作业} 
\chead{\Title}
\rhead{\Date}
\lfoot{\LaTeX}
\cfoot{\thepage~/~\pageref{LastPage}}
\fancypagestyle{plain}{
    \fancyhf{}
    \lfoot{\LaTeX}
    \cfoot{\thepage~/~\pageref{LastPage}}
    \renewcommand{\headrulewidth}{0pt}
}
\usepackage{caption}
\usepackage{graphicx}
\usepackage{subcaption}
\usepackage{float}
\usepackage{tikz}
\usepackage{indentfirst}
\usepackage{enumitem}
\setlength{\parindent}{2em}
% MINIPAGE USAGE
% \begin{figure}[H]
%     \centering
%     \begin{minipage}{0.24\linewidth}
%         \centering
%         \caption*{图1-5}
%         \includegraphics[scale=0.7]{1-5.png}
%     \end{minipage}
% \end{figure}
\newenvironment{Question}
{
    \paragraph{题目}
}
{
}
\newenvironment{Answer}
{
    \paragraph{解答}
}
{
}
\newenvironment{itemlist}{
    \begin{itemize}[itemsep=0pt,topsep=1em,parsep=0pt,leftmargin=2em,labelsep=1em,itemindent=2em] 
}{
    \end{itemize}
}
\newenvironment{numlist}{
    \begin{enumerate}[itemsep=0pt,topsep=1em,parsep=0pt,leftmargin=2em,labelsep=1em,itemindent=2em] 
}{
    \end{enumerate}
}
\title{\textbf{\Title}}
\author{\Serial~\Name\\\Academy}

\newcommand{\fileauthor}{Author}
\newcommand{\fileauthoracademy}{Academy}
\newcommand{\fileauthorclass}{Class}
\newcommand{\fileauthorserial}{54749110}

\begin{document}
\maketitle
\tableofcontents
\section{\TeX 模板说明}
本模板是一个\LaTeX 作业模板,包含\code{template.tex}、\code{config.tex}和\code{style.tex}三个文件。在\code{config.tex}中您可以修改文档的相关信息。

全局字体和公式字体已更换为Times New Roman。全文段落可以自动缩进,请不要使用\code{\\\\}(强制换行符号)而是使用自然换行(在\TeX 源文件的两个段落之间空一行),否则自动缩进将会失效。

请使用pdf\LaTeX 编译。生成标签和交叉引用需要编译两次,如果底部页码出现错误,请再次编译。
\subsection{列表功能测试}
这是正常的正文文本。在该\code{itemize}环境中,导入了\code{enumitem}宏包并将\code{itemsep}修改为0pt,\code{topsep}修改为0pt,\code{parsep}修改为0pt,\code{leftmargin}修改为2em。在该\code{itemize}环境中,导入了\code{enumitem}宏包并将\code{itemsep}修改为0pt,\code{topsep}修改为0pt,\code{parsep}修改为0pt,\code{leftmargin}修改为2em。
\begin{itemlist}
    \item 这是无序列表项目1。在该\code{itemize}环境中,导入了\code{enumitem}宏包并将\code{itemsep}修改为0pt,\code{topsep}修改为0pt,\code{parsep}修改为0pt,\code{leftmargin}修改为2em。
    \item 这是无序列表项目2。在该\code{itemize}环境中,导入了\code{enumitem}宏包并将\code{itemsep}修改为0pt,\code{topsep}修改为0pt,\code{parsep}修改为0pt,\code{leftmargin}修改为2em。
\end{itemlist}

这是正常的正文文本。在该\code{itemize}环境中,导入了\code{enumitem}宏包并将\code{itemsep}修改为0pt,\code{topsep}修改为0pt,\code{parsep}修改为0pt,\code{leftmargin}修改为2em。在该\code{itemize}环境中,导入了\code{enumitem}宏包并将\code{itemsep}修改为0pt,\code{topsep}修改为0pt,\code{parsep}修改为0pt,\code{leftmargin}修改为2em。

This is a formal paragraph.
\begin{numlist}
    \item 这是有序列表项目1。在该\code{itemize}环境中,导入了\code{enumitem}宏包并将\code{itemsep}修改为0pt,\code{topsep}修改为0pt,\code{parsep}修改为0pt,\code{leftmargin}修改为2em。
    \item 这是有序列表项目2。在该\code{itemize}环境中,导入了\code{enumitem}宏包并将\code{itemsep}修改为0pt,\code{topsep}修改为0pt,\code{parsep}修改为0pt,\code{leftmargin}修改为2em。
\end{numlist}
\subsection{Question和Answer环境测试}
\begin{Question}
    
This is a test paragraph with automatical indent.

This is another test paragraph with automatical indent.

\end{Question}
\begin{Answer}
    This is a test paragraph with automatical indent.
    
    This is another test paragraph with automatical indent.
\end{Answer}
\subsection{代码环境测试}
行内代码可以使用\code{\\code{}}表示。

以下是一个代码块示例:
\begin{lstlisting}[language=C++]
    #include<bits/stdc++.h>
    using namespace std;
    int main(){
        cout<<"Welcome to LaTeX!"<<endl;
        return 0;
    }
\end{lstlisting}
\subsection{公式测试}
这是一个\code{align}环境中的行间公式。
\begin{align}
    \int_a^bf(x)\mathrm{d}x=F(b)-F(a)
\end{align}

这是一个行内公式:$\sum_{k=1}^{10}k$。
\end{document}