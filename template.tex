\documentclass{homework}
\title{Test}
\begin{document}
\maketitle
% \tableofcontents
% \section{\TeX 模板说明}
% 本模板是一个\LaTeX 作业模板,包含\code{template.tex}、\code{config.tex}和\code{style.tex}三个文件。在\code{config.tex}中您可以修改文档的相关信息。

% 全局字体和公式字体已更换为Times New Roman。全文段落可以自动缩进,请不要使用\code{\\\\}(强制换行符号)而是使用自然换行(在\TeX 源文件的两个段落之间空一行),否则自动缩进将会失效。

% 请使用pdf\LaTeX 编译。生成标签和交叉引用需要编译两次,如果底部页码出现错误,请再次编译。
% 这是正常的正文文本。在该\code{itemize}环境中,导入了\code{enumitem}宏包并将\code{itemsep}修改为0pt,\code{topsep}修改为0pt,\code{parsep}修改为0pt,\code{leftmargin}修改为2em。在该\code{itemize}环境中,导入了\code{enumitem}宏包并将\code{itemsep}修改为0pt,\code{topsep}修改为0pt,\code{parsep}修改为0pt,\code{leftmargin}修改为2em。
% \begin{itemlist}
    % \item 这是无序列表项目1。在该\code{itemize}环境中,导入了\code{enumitem}宏包并将\code{itemsep}修改为0pt,\code{topsep}修改为0pt,\code{parsep}修改为0pt,\code{leftmargin}修改为2em。
    % \item 这是无序列表项目2。在该\code{itemize}环境中,导入了\code{enumitem}宏包并将\code{itemsep}修改为0pt,\code{topsep}修改为0pt,\code{parsep}修改为0pt,\code{leftmargin}修改为2em。
% \end{itemlist}

% 这是正常的正文文本。在该\code{itemize}环境中,导入了\code{enumitem}宏包并将\code{itemsep}修改为0pt,\code{topsep}修改为0pt,\code{parsep}修改为0pt,\code{leftmargin}修改为2em。在该\code{itemize}环境中,导入了\code{enumitem}宏包并将\code{itemsep}修改为0pt,\code{topsep}修改为0pt,\code{parsep}修改为0pt,\code{leftmargin}修改为2em。

This is a formal paragraph.
% \begin{numlist}
    % \item 这是有序列表项目1。在该\code{itemize}环境中,导入了\code{enumitem}宏包并将\code{itemsep}修改为0pt,\code{topsep}修改为0pt,\code{parsep}修改为0pt,\code{leftmargin}修改为2em。
    % \item 这是有序列表项目2。在该\code{itemize}环境中,导入了\code{enumitem}宏包并将\code{itemsep}修改为0pt,\code{topsep}修改为0pt,\code{parsep}修改为0pt,\code{leftmargin}修改为2em。
% \end{numlist}
% \subsection{Question和Answer环境测试}
% \begin{Question}
    
% This is a test paragraph with automatical indent.

% This is another test paragraph with automatical indent.

% \end{Question}
% \begin{Answer}
    This is a test paragraph with automatical indent.
    
    This is another test paragraph with automatical indent.
% \end{Answer}
% \subsection{代码环境测试}
% 行内代码可以使用\code{\\code{}}表示。

% 以下是一个代码块示例:
% \begin{lstlisting}[language=C++]
    % #include<bits/stdc++.h>
    % using namespace std;
    % int main(){
    %     cout<<"Welcome to LaTeX!"<<endl;
    %     return 0;
    % }
% \end{lstlisting}
% \subsection{公式测试}
% 这是一个\code{align}环境中的行间公式。
\begin{align}
     \int_a^bf(x)\mathrm{d}x=F(b)-F(a)
\end{align}

$\sum_{k=1}^{10}k$
\end{document}